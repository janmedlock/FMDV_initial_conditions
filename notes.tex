\documentclass{jpmarticle}

\usepackage{units}
\usepackage{csquotes}
\usepackage[style=vancouver]{biblatex}
\addbibresource{notes.bib}

\title{Crank--Nicolson for some SIR models}
\author{Jan Medlock}

\begin{document}

\maketitle

\section{Unstructured SIR model}

The model is
\begin{equation}
  \label{model_unstructured}
  \begin{split}
    \frac{\md S}{\md t}(t) &=
    b(t) N(t) - \beta I(t) S(t) - \mu S(t),
    \\
    \frac{\md I}{\md t}(t) &=
    \beta I(t) S(t) - \gamma I(t) - \mu I(t),
    \\
    \frac{\md R}{\md t}(t) &=
    \gamma I(t) - \mu R(t),
  \end{split}
\end{equation}
with birth rate
\begin{equation}
  b(t) = \mu \left[
    1 + \sigma \sqrt{2} \cos\left(2 \pi t\right)
  \right],
\end{equation}
and population size
\begin{equation}
  N(t) = S(t) + I(t) + R(t).
\end{equation}

The total population follows
\begin{equation}
  \begin{split}
    \frac{\md N}{\md t}(t)
    &= b(t) N(t) - \mu N(t)
    \\
    &= \mu \sigma \sqrt{2} \cos\left(2 \pi t\right) N(t),
  \end{split}
\end{equation}
so that
\begin{equation}
  \log N(t) = \log N(0)
  + \frac{\mu \sigma}{\pi \sqrt{2}}
  \sin\left(2 \pi t\right),
\end{equation}
or
\begin{equation}
  N(t) = N(0) \exp\left[
    \frac{\mu \sigma}{\pi \sqrt{2}}
    \sin\left(2 \pi t\right)
  \right],
\end{equation}
i.e.~periodic with period $1$.

For the system of differential equations
\begin{equation}
  \begin{split}
    \frac{\md \vec{Y}}{\md t}(t)
    &= \vec{f}\left(t, \vec{Y}(t)\right),
    \\
    \vec{Y}(0) &= \vec{Y}^0,
  \end{split}
\end{equation}
and time step $\Delta t$, let $t^k = k \Delta t$ for
$k \in \{0, 1, \ldots, K - 1\}$ and
\begin{equation}
  \vec{Y}^k \approx \vec{Y}(t^k).
\end{equation}
The Crank--Nicolson scheme is
\begin{equation}
  \frac{\vec{Y}^k - \vec{Y}^{k - 1}}{\Delta t}
  = \frac{
    \vec{f}\left(t^k, \vec{Y}^k\right)
    + \vec{f}\left(t^{k - 1}, \vec{Y}^{k - 1}\right)
  }{2}
\end{equation}
for $k \in \{1, \ldots, K - 1\}$.
That is, we must find $\vec{Y}^k$ such that
\begin{equation}
  \label{cn_unstructured}
  \vec{Y}^k
  - \frac{\Delta t}{2}
  \vec{f}\left(t^k, \vec{Y}^k\right)
  - \vec{Y}^{k - 1}
  - \frac{\Delta t}{2}
  \vec{f}\left(t^{k - 1}, \vec{Y}^{k - 1}\right)
  = \vec{0}.
\end{equation}
If $\vec{f}\left(t^k, \vec{Y}^k\right)$ is linear in $\vec{Y}^k$
then equation \eqref{cn_unstructured} may be solved efficiently for
$\vec{Y}^k$ using matrix algebra, but generally we must use a
nonlinear solver.

For model \eqref{model_unstructured},
\begin{equation}
  \vec{Y}^k =
  \begin{bmatrix}
    S^k \\ I^k \\ R^k
  \end{bmatrix}
\end{equation}
and
\begin{equation}
  \vec{f}\left(t^k, \vec{Y}^k\right) =
  \begin{bmatrix}
    b\left(t^k\right) N^k
    - \beta I^k S^k
    - \mu S^k
    \\
    \beta I^k S^k
    - \gamma I^k
    - \mu I^k
    \\
    \gamma I^k
    - \mu R^k
  \end{bmatrix},
\end{equation}
where
\begin{equation}
  \label{eq:1}
  N^k = S^k + I^k + R^k.
\end{equation}


\section{Age-structured SIR model}

The model is
\begin{equation}
  \label{model_age_structured}
  \begin{split}
    s(t, \unit[0]{y})
    &= b(t) \int_0^\infty m(a) n(t, a) \md a,
    \\
    \frac{\partial s}{\partial t}(t, a)
    + \frac{\partial s}{\partial a}(t, a)
    &= - \lambda(t) s(t, a) - \mu(a) s(t, a),
    \\
    i(t, \unit[0]{y})
    &= 0,
    \\
    \frac{\partial i}{\partial t}(t, a)
    + \frac{\partial i}{\partial a}(t, a)
    &= \lambda(t) s(t, a) - \gamma i(t, a) - \mu(a) i(t, a),
    \\
    r(t, \unit[0]{y})
    &= 0,
    \\
    \frac{\partial r}{\partial t}(t, a)
    + \frac{\partial r}{\partial a}(t, a)
    &= \gamma i(t, a) - \mu(a) r(t, a),
  \end{split}
\end{equation}
with birth rate
\begin{equation}
  b(t) = \bar{b} \left[
    1 + \sigma \sqrt{2} \cos\left(2 \pi t\right)
  \right],
\end{equation}
force of infection
\begin{equation}
  \lambda(t) = \int_0^\infty \beta(a) i(t, a) \md a,
\end{equation}
and population size
\begin{equation}
  n(t, a) = s(t, a) + i(t, a) + r(t, a).
\end{equation}

\subsection{Total population}

The total population follows the linear PDE
\begin{equation}
  \begin{split}
    n(t, \unit[0]{y})
    &= b(t) \int_0^\infty m(a) n(t, a) \md a,
    \\
    \frac{\partial n}{\partial t}(t, a)
    + \frac{\partial n}{\partial a}(t, a)
    &= - \mu(a) n(t, a).
  \end{split}
\end{equation}

Because $b(t)$ is periodic with period $T = \unit[1]{y}$, we used
Floquet theory \autocite{parker_1992} to find the stable age
distribution and the asymptotic population growth rate. Floquet theory
requires the fundamental solution $\Phi(t, a, a')$ to
\begin{equation}
  \begin{split}
    \Phi(t, \unit[0]{y}, a')
    &= b(t) \int_0^\infty m(a) \Phi(t, a, a') \md a,
    \\
    \frac{\partial \Phi}{\partial t}(t, a, a')
    + \frac{\partial \Phi}{\partial a}(t, a, a')
    &= - \mu(a) \Phi(t, a, a'),
    \\
    \Phi(t_0, a, a')
    &= \delta(a - a'),
  \end{split}
\end{equation}
where $\delta(x)$ is the Dirac delta.

To solve this numerically, we used the Crank--Nicolson method on
characteristics and the composite trapezoid rule for the birth
integral \autocite{milner_1992}.  Given the time step $\Delta t$,
let $a_i = i \Delta t$
and $a'_j = j \Delta t$
for $i, j \in \{0, 1, 2, \ldots, I - 1\}$;
$t^k = t_0 + k \Delta t$
for $k \in \{0, 1, \ldots, K - 1\}$;
and $\Phi_{i, j}^k \approx \Phi(t^k, a_i, a'_j)$.
For each $j$
and each $k \geq 1$, the Crank--Nicolson method on characteristics is
\begin{equation}
  \label{CN_step}
  \frac{\Phi_{i, j}^k - \Phi_{i - 1, j}^{k - 1}}{\Delta t}
  = - \mu(a_{i - 1 / 2})
  \frac{\Phi_{i, j}^k + \Phi_{i - 1, j}^{k - 1}}{2},
\end{equation}
or
\begin{equation}
  \Phi_{i, j}^k
  = \frac{1 - C_{i - 1 / 2}}{1 + C_{i - 1 / 2}}
  \Phi_{i - 1, j}^{k - 1},
\end{equation}
with
\begin{equation}
  C_{i - 1 / 2}
  = \frac{1}{2} \mu(a_{i - 1 / 2}) \Delta t,
\end{equation}
for $i \in \{1, 2, \ldots, I - 2\}$.  For $i = I - 1$,
a term was added to prevent aging out of this
last age group:
\begin{equation}
  \Phi_{I - 1, j}^k
  = \frac{1 - C_{I - 3 / 2}}{1 + C_{I - 3 / 2}}
  \Phi_{I - 2, j}^{k - 1}
  + \frac{1 - C_{I - 1}}{1 + C_{I - 1}}
  \Phi_{I - 1, j}^{k - 1},
\end{equation}
with
\begin{equation}
  C_{I - 1}
  = \frac{1}{2} \mu(a_{I - 1}) \Delta t.
\end{equation}
For $i = 0$, the birth integral is given by the composite trapezoid rule,
\begin{equation}
  \label{birth_step}
  \Phi_{0, j}^k =
  b(t^k)
  \sum_{i = 1}^{I - 1}
  \frac{m(a_i) \Phi_{i, j}^k
    + m(a_{i - 1}) \Phi_{i - 1, j}^k}{2}
  \Delta t,
\end{equation}
or
\begin{equation}
  \mat{\Phi}_0^k = b(t^k) \vec{v} \mat{\Phi}^k,
\end{equation}
with the vector $\vec{v} = [v_i]$ for
\begin{equation}
  v_i =
  \begin{cases}
    \frac{1}{2} m(a_i) \Delta t
    & \text{if $i = 0$ or $i = I - 1$}, \\
    m(a_i) \Delta t
    & \text{otherwise}.
  \end{cases}
\end{equation}
The initial condition is
\begin{equation}
  \Phi_{i, j}^0 =
  \begin{cases}
    1 & \text{if $i = j$}, \\
    0 & \text{otherwise}.
  \end{cases}
\end{equation}
Considering $\mat{\Phi}^k = [\Phi_{i, j}^k]$ as a matrix that
evolves in time, the method is easily implemented with matrix algebra:
the Crank--Nicolson step \eqref{CN_step} is
\begin{equation}
  \mat{\Phi}^k = \mat{M} \mat{\Phi}^{k - 1},
\end{equation}
with the matrix $\mat{M} = [M_{i, j}]$ where
\begin{equation}
  M_{i, j} =
  \begin{cases}
    \frac{1 - C_{i - 1 / 2}}{1 + C_{i - 1 / 2}}
    & \text{if $i = j + 1$}, \\
    \frac{1 - C_{I - 1}}{1 + C_{I - 1}} & \text{if $i = j = I - 1$}, \\
    0 & \text{otherwise}.
  \end{cases}
\end{equation}
The initial condition is
\begin{equation}
  \mat{\Phi}^0 = \mat{I},
\end{equation}
where $\mat{I}$ is the $I \times I$ identity matrix.

Using this numerical scheme, we solved for the monodromy matrix, the
fundamental solution after one period:
\begin{equation}
  \mat{\Psi} = [\Psi_{i, j}] \approx [\Phi(t_0 + T, a_i, a'_j)].
\end{equation}
The monodromy matrix projects the population forward at by one period,
\begin{equation}
  \vec{n}(t_0 + T) = \mat{\Psi} \vec{n}(t_0),
\end{equation}
where $\vec{n}(t) = [n(t, a_i)]$.
Using the monodromy matrix to repeatedly project the population
forward gives
\begin{equation}
  \vec{n}(t_0 + K T)
  = \mat{\Psi}^K \vec{n}(t_0)
  \to \rho_0^K \vec{w}_0
  = \me^{r K T} \vec{w}_0
\end{equation}
as $K \to \infty$,
where $\rho_0$ is the dominant eigenvalue, i.e.~the eigenvalue with
largest magnitude, of $\mat{\Psi}$;
the corresponding right eigenvector $\vec{w}_0$ is the stable age
distribution; and
\begin{equation}
  r = \frac{1}{T} \log \rho_0
\end{equation}
is the asymptotic population growth rate.

We used a nonlinear solver to search for the mean birth rate $\bar{b}$
that gives population growth rate $r = 0$.

\printbibliography

\end{document}
